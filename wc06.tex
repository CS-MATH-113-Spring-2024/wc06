\documentclass[a4paper]{exam}

\usepackage{amsmath}
\usepackage[a4paper]{geometry}

\usepackage{draftwatermark}
\SetWatermarkText{Sample Solution}
\SetWatermarkScale{3}
\printanswers


\header{CS/MATH 113}{WC06: Logical Inference}{Spring 2024}
\footer{}{Page \thepage\ of \numpages}{}
\runningheadrule
\runningfootrule

\printanswers

\newcommand\cb{\color{blue}}
\newcommand\cre{\color{red}}
\newcommand\cg{\color{green}}
\newcommand\cy{\color{yellow}}
\newcommand\cgr{\color{gray}}

\title{Weekly Challenge 06: Logical Inference}
\author{CS/MATH 113 Discrete Mathematics}
\date{Spring 2024}

\qformat{{\large\bf \thequestion. \thequestiontitle}\hfill}
\boxedpoints

\begin{document}
\maketitle

\begin{questions}

  \titledquestion{Sacred Secrets}[5] Legend has it that one of the TAs from a previous offering of CS/MATH 113 prepared a manual, ``Sacred Secrets: How to Earn an A+ and Stay Sane''. However their own sanity is sadly no longer intact. The \LaTeX\ source and the repository got deleted and all that exists about the location of the only printed copy are the following instructions.
  \begin{enumerate}
  \item There is a hint at Learn Courtyard or at the Gym.
  \item If your TA is sitting in Ehsas or they are absent, then there is a hint at Learn Courtyard.
  \item If your TA is not sitting in Ehsaas, then there is a hint at the Gym.
  \item If there are people in Learn Courtyard, then there is no hint at Learn Courtyard.
  \item If there is a hint at Learn Courtyard, then the manual is at Zen Garden.
  \item If there is hint at the Gym, then the manual is at Earth Courtyard.
  \item If your TA is absent, then the manual is at Fire Courtyard.
  \end{enumerate}
  You notice that there are people in Learn Courtyard. Show how you can infer the location of the manual.

  \begin{solution}
    Let us use the following propositions.\\
    \begin{tabular}{l@{ : }l}
      $learn$ & there is a hint at Learn Courtyard\\
      $gym$ & there is a hint at the Gym\\
      $ehsas$ & your TA is sitting in Ehsas\\
      $absent$ & your TA is absent\\
      $people$ & there are people in Learn Courtyard\\
      $zen$ & the manual is at Zen Garden\\
      $earth$ & the manual is at Earth Courtyard\\
      $fire$ & the manual is at Fire Courtyard
    \end{tabular}

    Then the clues and observations are:
    \begin{align}
      learn & \lor gym \label{c1}\\
      ehsas\lor absent & \implies learn \label{c2}\\
      \neg ehsas & \implies gym \label{c3}\\
      people & \implies \neg learn \label{c4}\\
      learn & \implies zen \label{c5}\\
      gym & \implies earth \label{c6}\\
      absent & \implies fire \label{c7}\\
      people \label{c8}
    \end{align}

    and we can reason as follows:
    \begin{align}
      \neg learn && \text{modus ponens on (\ref{c4}) and (\ref{c8})} \label{c9}\\
      gym && \text{disjunctive syllogism on (\ref{c1}) and (\ref{c9})} \label{c10}\\
      earth && \text{modus ponens on (\ref{c6}) and (\ref{c10})} \label{c11}
    \end{align}
    Therefore, the manual is at Earth Courtyard.
  \end{solution}

\titledquestion{The Other Side} You are given four cards each of which has a number on one side and a letter on another. You place them on a table in front of you and the four cards read: \texttt{A 5 2 J}. Which cards would you turn over in order to test the following rule? Explain your choice.
      {\quotation Cards with 5 on one side have J on the other side.}

  \begin{solution}
    The rule can be written as: number side is 5 $\implies$ letter side is J.

    To test the rule, we simply need to check cards that match two situations:
    \begin{enumerate}
    \item\label{ant} whenever the antecedent is true, the consequent must be true.
    \item\label{con} whenever the consequent is false, the antecedent must be false.
    \end{enumerate}

    Let us see how the cards match these situations.

    \begin{description}
    \item[A] This matches condition \ref{con}. We will turn over the card. If the number on the other side is 5, the rule is broken. If the number is different from 5, the rule holds.
    \item[5] This matches condition \ref{ant}. We will turn over the card. If the letter on the other side is J, the rule holds. If the letter is different from J, the rule is broken. 
    \item[2] This does not match any of the conditions. There is no use turning this card. There is no bearing on the rule if the letter on the other side is J or not.
    \item[J] This does not match any of the conditions. There is no use turning this card. There is no bearing on the rule if the number on the other side is 5 or not.
    \end{description}
    \end{solution}

  
\end{questions}

\end{document}

%%% Local Variables:
%%% mode: latex
%%% TeX-master: t
%%% End:
